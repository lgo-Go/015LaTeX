\documentclass{article}
\usepackage[utf8]{inputenc}
\usepackage[russian]{babel}
\usepackage[left=0mm, top=10mm, right=10mm, bottom=10mm, nohead, nofoot]{geometry}
\usepackage{amsmath}
\def\dd#1#2{\dfrac{\partial#1}{\partial#2}}
\def\ddd#1#2{\dfrac{\partial^2#1}{\partial#2}}

\begin{document}

    \hspace{10mm} {\LARGE \bf{17 Уравнений, Которые Изменили Мир}}

    \hspace{40mm} {\LARGE по версии Иэна Стюарта}

    \vspace{10mm}
    \begin{tabular}{l l l l}

        \vspace{5mm}
            1. & 
            \textbf{Теорема Пифагора} & 
            $a^2+b^2=c^2$ & 
            Пифагор, 530 до .н.э. \\
    
        \vspace{5mm}
            2. & 
            \textbf{Логарифмы} & 
            $\log{xy}=\log{x}+\log{y}$ & 
            Джон Непер, 1610 \\
    
        \vspace{5mm}
            3. & 
            \textbf{Производная} & 
            $\dfrac {df}{dt} = \lim\limits_{h \to 0} \dfrac{f(t+h)-f(t)}{h}$ & 
            Ньютон, 1668  \\
    
        \vspace{5mm}
            4. &
            \textbf{Закон Всемирного Тяготения} & 
            $F=G \dfrac {m_1m_2}{r^2}$ & 
            Ньютон, 1687\\
    
        \vspace{5mm}
            5. &
            \textbf{Квадратный Корень из Минус Единицы} &
            $i^2=-1$ & Эйлер, 1750\\
    
        \vspace{5mm}
            6. & 
            \textbf{Формула Эйлера для Многогранника} & 
            $V-E+F=2$ & 
            Эйлер, 1751 \\
    
        \vspace{5mm}
            7. & 
            \textbf{Нормальное Распределение} 
            & $\Phi(x) = \dfrac{1}{\sqrt{2\pi\rho}} e^{\frac{(x-\rho)^2}{2\rho^2}}$ & 
            К.\,Ф.\,Гаусс \\
        
        \vspace{5mm}
            8. & 
            \textbf{Волновое Уравнение} & 
            $\ddd{u}{t^2}=c^2\ddd{u}{x^2}$ & 
            Ж.\,Д'Алмбер \\
    
        \vspace{5mm}
            9. & 
            \textbf{Преобразование Фурье} & 
            $f(\omega)=\displaystyle \int\limits_\infty^\infty f(x)e^{-2\pi i x\omega} \text{d}\,x$ & 
            Ш.\,Фурье, 1822 \\
    
        \vspace{5mm}
            10. & 
            \textbf{Уравнение Навье-Стокса}& 
            $\rho \left(\dd{\bf v}t + \bf v \cdot \nabla \bf v \right) = -\nabla p + \nabla \cdot \bf T + \bf f$ & 
            А.\,Навье, Д.\,Стокс, 1845 \\
    
        \vspace{5mm}
            11. & 
            \textbf{Уравнения Максвелла} & 
            \begin{tabular}{l l}
                $\nabla \cdot {\bf E} = \frac{\rho}{\varepsilon_\scriptscriptstyle 0}$ & $\nabla \cdot {\bf H} = 0$ \\
                $\nabla \times {\bf E} = -\frac1c \frac{\partial {\bf H}}{\partial t}$ & $\nabla \times {\bf H} = \frac1c \frac{\partial E}{\partial t}$ 
            \end{tabular} & 
            Д.\,К.\,Максвелл, 1865 \\
    
        \vspace{5mm}
            12. & 
            \textbf{Второй Закон Термодинамики} &
            d$S \ge 0$ & 
            Л.\,Больцман, 1874 \\
    
        \vspace{5mm}
            13. & 
            \textbf{Теория Относительности} & 
            $E = m c^2$ & 
            Энштейн, 1905\\
    
        \vspace{5mm}
            14. & 
            \textbf{Уравнение Шрёдингера} & 
            $i h \dfrac{\partial}{\partial t} \Psi = H \Psi$ & 
            Э.\,Шрёдингер, 1927 \\
    
        \vspace{5mm}
            15. & 
            \textbf{Информационная Теория} &
            $H = \displaystyle -\sum p(x) \log p(x)$ & 
            К.\,Шенон, 1949 \\
    
        \vspace{5mm}
            16. & 
            \textbf{Теория Хаоса} & 
            $x_{t+1} = k x_t (1 - x_t)$ & 
            Роберт Мэй, 1975 \\
    
        \vspace{5mm}
            17. & 
            \textbf{Уравнение Блэка-Шоулза} & 
            $\dfrac12 \sigma^2 S^2 \ddd{V}{S^2} + r S \dd{V}{S} + \dd{V}{t} - r V = 0$ & 
            Ф.\,Блэк, М.\,Шоулз, 1990 \\
    
    \end{tabular}

\end{document}
